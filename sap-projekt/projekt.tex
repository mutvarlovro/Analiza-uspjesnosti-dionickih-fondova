% Options for packages loaded elsewhere
\PassOptionsToPackage{unicode}{hyperref}
\PassOptionsToPackage{hyphens}{url}
%
\documentclass[
]{article}
\usepackage{lmodern}
\usepackage{amssymb,amsmath}
\usepackage{ifxetex,ifluatex}
\ifnum 0\ifxetex 1\fi\ifluatex 1\fi=0 % if pdftex
  \usepackage[T1]{fontenc}
  \usepackage[utf8]{inputenc}
  \usepackage{textcomp} % provide euro and other symbols
\else % if luatex or xetex
  \usepackage{unicode-math}
  \defaultfontfeatures{Scale=MatchLowercase}
  \defaultfontfeatures[\rmfamily]{Ligatures=TeX,Scale=1}
\fi
% Use upquote if available, for straight quotes in verbatim environments
\IfFileExists{upquote.sty}{\usepackage{upquote}}{}
\IfFileExists{microtype.sty}{% use microtype if available
  \usepackage[]{microtype}
  \UseMicrotypeSet[protrusion]{basicmath} % disable protrusion for tt fonts
}{}
\makeatletter
\@ifundefined{KOMAClassName}{% if non-KOMA class
  \IfFileExists{parskip.sty}{%
    \usepackage{parskip}
  }{% else
    \setlength{\parindent}{0pt}
    \setlength{\parskip}{6pt plus 2pt minus 1pt}}
}{% if KOMA class
  \KOMAoptions{parskip=half}}
\makeatother
\usepackage{xcolor}
\IfFileExists{xurl.sty}{\usepackage{xurl}}{} % add URL line breaks if available
\IfFileExists{bookmark.sty}{\usepackage{bookmark}}{\usepackage{hyperref}}
\hypersetup{
  pdftitle={Analiza uspješnosti dioničkih fondova},
  pdfauthor={Sap-projekt},
  hidelinks,
  pdfcreator={LaTeX via pandoc}}
\urlstyle{same} % disable monospaced font for URLs
\usepackage[margin=1in]{geometry}
\usepackage{color}
\usepackage{fancyvrb}
\newcommand{\VerbBar}{|}
\newcommand{\VERB}{\Verb[commandchars=\\\{\}]}
\DefineVerbatimEnvironment{Highlighting}{Verbatim}{commandchars=\\\{\}}
% Add ',fontsize=\small' for more characters per line
\usepackage{framed}
\definecolor{shadecolor}{RGB}{248,248,248}
\newenvironment{Shaded}{\begin{snugshade}}{\end{snugshade}}
\newcommand{\AlertTok}[1]{\textcolor[rgb]{0.94,0.16,0.16}{#1}}
\newcommand{\AnnotationTok}[1]{\textcolor[rgb]{0.56,0.35,0.01}{\textbf{\textit{#1}}}}
\newcommand{\AttributeTok}[1]{\textcolor[rgb]{0.77,0.63,0.00}{#1}}
\newcommand{\BaseNTok}[1]{\textcolor[rgb]{0.00,0.00,0.81}{#1}}
\newcommand{\BuiltInTok}[1]{#1}
\newcommand{\CharTok}[1]{\textcolor[rgb]{0.31,0.60,0.02}{#1}}
\newcommand{\CommentTok}[1]{\textcolor[rgb]{0.56,0.35,0.01}{\textit{#1}}}
\newcommand{\CommentVarTok}[1]{\textcolor[rgb]{0.56,0.35,0.01}{\textbf{\textit{#1}}}}
\newcommand{\ConstantTok}[1]{\textcolor[rgb]{0.00,0.00,0.00}{#1}}
\newcommand{\ControlFlowTok}[1]{\textcolor[rgb]{0.13,0.29,0.53}{\textbf{#1}}}
\newcommand{\DataTypeTok}[1]{\textcolor[rgb]{0.13,0.29,0.53}{#1}}
\newcommand{\DecValTok}[1]{\textcolor[rgb]{0.00,0.00,0.81}{#1}}
\newcommand{\DocumentationTok}[1]{\textcolor[rgb]{0.56,0.35,0.01}{\textbf{\textit{#1}}}}
\newcommand{\ErrorTok}[1]{\textcolor[rgb]{0.64,0.00,0.00}{\textbf{#1}}}
\newcommand{\ExtensionTok}[1]{#1}
\newcommand{\FloatTok}[1]{\textcolor[rgb]{0.00,0.00,0.81}{#1}}
\newcommand{\FunctionTok}[1]{\textcolor[rgb]{0.00,0.00,0.00}{#1}}
\newcommand{\ImportTok}[1]{#1}
\newcommand{\InformationTok}[1]{\textcolor[rgb]{0.56,0.35,0.01}{\textbf{\textit{#1}}}}
\newcommand{\KeywordTok}[1]{\textcolor[rgb]{0.13,0.29,0.53}{\textbf{#1}}}
\newcommand{\NormalTok}[1]{#1}
\newcommand{\OperatorTok}[1]{\textcolor[rgb]{0.81,0.36,0.00}{\textbf{#1}}}
\newcommand{\OtherTok}[1]{\textcolor[rgb]{0.56,0.35,0.01}{#1}}
\newcommand{\PreprocessorTok}[1]{\textcolor[rgb]{0.56,0.35,0.01}{\textit{#1}}}
\newcommand{\RegionMarkerTok}[1]{#1}
\newcommand{\SpecialCharTok}[1]{\textcolor[rgb]{0.00,0.00,0.00}{#1}}
\newcommand{\SpecialStringTok}[1]{\textcolor[rgb]{0.31,0.60,0.02}{#1}}
\newcommand{\StringTok}[1]{\textcolor[rgb]{0.31,0.60,0.02}{#1}}
\newcommand{\VariableTok}[1]{\textcolor[rgb]{0.00,0.00,0.00}{#1}}
\newcommand{\VerbatimStringTok}[1]{\textcolor[rgb]{0.31,0.60,0.02}{#1}}
\newcommand{\WarningTok}[1]{\textcolor[rgb]{0.56,0.35,0.01}{\textbf{\textit{#1}}}}
\usepackage{graphicx,grffile}
\makeatletter
\def\maxwidth{\ifdim\Gin@nat@width>\linewidth\linewidth\else\Gin@nat@width\fi}
\def\maxheight{\ifdim\Gin@nat@height>\textheight\textheight\else\Gin@nat@height\fi}
\makeatother
% Scale images if necessary, so that they will not overflow the page
% margins by default, and it is still possible to overwrite the defaults
% using explicit options in \includegraphics[width, height, ...]{}
\setkeys{Gin}{width=\maxwidth,height=\maxheight,keepaspectratio}
% Set default figure placement to htbp
\makeatletter
\def\fps@figure{htbp}
\makeatother
\setlength{\emergencystretch}{3em} % prevent overfull lines
\providecommand{\tightlist}{%
  \setlength{\itemsep}{0pt}\setlength{\parskip}{0pt}}
\setcounter{secnumdepth}{-\maxdimen} % remove section numbering

\title{Analiza uspješnosti dioničkih fondova}
\author{Sap-projekt}
\date{17.01.2021.}

\begin{document}
\maketitle

\hypertarget{analiza-uspjeux161nosti-dioniux10dkih-fondova}{%
\section{Analiza uspješnosti dioničkih
fondova}\label{analiza-uspjeux161nosti-dioniux10dkih-fondova}}

\hypertarget{opis-projekta}{%
\subsection{Opis projekta:}\label{opis-projekta}}

Ovaj projekt obavezni je dio izbornog kolegija Statistička analiza
podataka Fakulteta elektrotehnike i računarstva. Projekt je poslužio
primjeni teorijskih temelja stečenih na predavanjima na skup podataka iz
stvarnog svijeta. Kao pomoć u izradi projekta korišten je programski
jezik R koji je pružio potporu za izvođenje testiranja i bolju
vizualizaciju podataka.

\hypertarget{opis-problema}{%
\subsection{Opis problema:}\label{opis-problema}}

Rezultati istraživanja provođenih zadnih nekoliko desetljeća upućuju na
to da je gotovo nemoguće dugoročno i konzistentno ``pobijediti tržište''
biranjem dionica u portfelju, u odnosu na dobro diverzificirani tržišni
indeks. Cilj ovog projektnog zadatka je analizirati podatke o povratima
i strukturi dioničkih fondova, te analizirati koliko su uspješni unutar
svoje kategorije i u odnosu na tržište. Pritom je fokus ovog projekta
isključivo na tzv. otvorene fondove (eng. mutual fund) koji ulažu u
dionice i dostupni su malim investitorima.

\hypertarget{skup-podataka}{%
\subsection{Skup podataka:}\label{skup-podataka}}

Korišteni skup podataka sastoji se od velikog broja dioničkih fondova i
izraženi su u američkim dolarima. U daljnjem tekstu često će se
pojavljivati izraz ``uspješnost fonda''. Kao ocjena uspješnosti fonda
korišteni su srednji povrat fonda u razdoblju od deset godina i povrat
fonda nakon od 10 godina. Iako je uspješnost moguće definirati na više
načina, ovaj je odabran kao standardan način prilikom početka rada na
projektu te je kao takav zadržan.

\hypertarget{deskriptivna-statistika-skupa-podataka}{%
\subsection{Deskriptivna statistika skupa
podataka:}\label{deskriptivna-statistika-skupa-podataka}}

\begin{verbatim}
## -- Attaching packages --------------------------------------- tidyverse 1.3.0 --
\end{verbatim}

\begin{verbatim}
## v ggplot2 3.3.3     v purrr   0.3.4
## v tibble  3.0.4     v dplyr   1.0.2
## v tidyr   1.1.2     v stringr 1.4.0
## v readr   1.4.0     v forcats 0.5.0
\end{verbatim}

\begin{verbatim}
## -- Conflicts ------------------------------------------ tidyverse_conflicts() --
## x dplyr::filter() masks stats::filter()
## x dplyr::lag()    masks stats::lag()
\end{verbatim}

Premda da imamo podatke i o srednjem prošlogodišnjem povratu te
povratima za zadnjih 3 i 5 godina, povrati za 10 godina nam najbolje
pokazuju koliko je određeni fond pouzdan/dobar/uspješan jer nam daju
podatke kroz najduži period. Na sljedećem box plot dijagramu prikazani
su srednji povrati za 3, 5 i 10 godina. \#Vidimo da su u prosjeku
povrati za 10 godina najveći.

\begin{Shaded}
\begin{Highlighting}[]
\NormalTok{success10 =}\StringTok{ }\NormalTok{data}\OperatorTok{$}\NormalTok{fund_mean_annual_return_10years}
\NormalTok{success5 =}\StringTok{ }\NormalTok{data}\OperatorTok{$}\NormalTok{fund_mean_annual_return_5years}
\NormalTok{success3 =}\StringTok{ }\NormalTok{data}\OperatorTok{$}\NormalTok{fund_mean_annual_return_3years}
\KeywordTok{boxplot}\NormalTok{(success10, success5, success3, }\DataTypeTok{main=}\StringTok{"Fund mean annual return"}\NormalTok{, }
        \DataTypeTok{names=}\KeywordTok{c}\NormalTok{(}\StringTok{"10 yrs"}\NormalTok{, }\StringTok{"5 yrs"}\NormalTok{, }\StringTok{"3 yrs"}\NormalTok{), }\DataTypeTok{ylab=}\StringTok{"Percentage"}\NormalTok{)}
\end{Highlighting}
\end{Shaded}

\includegraphics{projekt_files/figure-latex/unnamed-chunk-2-1.pdf}

Sljedeći dijagram prikazuje udjele određenih stilova investiranja
fondova. U podatcima koje koristimo fondovi imaju 3 različita stila
investiranja; Growth, Value i Blend. Growth fondovi ulažu u tvrtke za
koje se očekuje da će brže rasti te su fondovi stoga riskantniji. Value
fondovi pronalaze trenutno podcjenjene tvrtke te u njih ulažu. Value
fondovi su po prirodi manje riskantni od growth fondova. Blend fondovi
kombiniraju value i growth stilove investiranja.

\begin{Shaded}
\begin{Highlighting}[]
\NormalTok{investment =}\StringTok{ }\NormalTok{data}\OperatorTok{$}\NormalTok{investment[data}\OperatorTok{$}\NormalTok{investment }\OperatorTok{!=}\StringTok{ "<undefined>"}\NormalTok{]}
\NormalTok{growth.number =}\StringTok{ }\KeywordTok{sum}\NormalTok{(investment }\OperatorTok{==}\StringTok{ "Growth"}\NormalTok{)}
\NormalTok{value.number =}\StringTok{ }\KeywordTok{sum}\NormalTok{(investment }\OperatorTok{==}\StringTok{ "Value"}\NormalTok{)}
\NormalTok{blend.number =}\StringTok{ }\KeywordTok{sum}\NormalTok{(investment }\OperatorTok{==}\StringTok{ "Blend"}\NormalTok{)}
\NormalTok{values =}\StringTok{ }\KeywordTok{c}\NormalTok{(growth.number, value.number, blend.number)}
\NormalTok{labels =}\StringTok{ }\KeywordTok{c}\NormalTok{(}\StringTok{"Growth"}\NormalTok{, }\StringTok{"Value"}\NormalTok{, }\StringTok{"Blend"}\NormalTok{)}
\NormalTok{pct =}\StringTok{ }\KeywordTok{round}\NormalTok{(values}\OperatorTok{/}\KeywordTok{sum}\NormalTok{(values)}\OperatorTok{*}\DecValTok{100}\NormalTok{, }\DataTypeTok{digits =} \DecValTok{2}\NormalTok{)}
\NormalTok{labels =}\StringTok{ }\KeywordTok{paste}\NormalTok{(labels, pct)}
\NormalTok{labels =}\StringTok{ }\KeywordTok{paste}\NormalTok{(labels,}\StringTok{"%"}\NormalTok{)}
\KeywordTok{pie}\NormalTok{(values, }\DataTypeTok{labels=}\NormalTok{labels, }\DataTypeTok{col=}\KeywordTok{rainbow}\NormalTok{(}\KeywordTok{length}\NormalTok{(labels)))}
\end{Highlighting}
\end{Shaded}

\includegraphics{projekt_files/figure-latex/unnamed-chunk-3-1.pdf}

Slijedeći dijagram prikazuje udjele imovine fonda u pojedinim sektorima.

\begin{Shaded}
\begin{Highlighting}[]
\NormalTok{materials =}\StringTok{ }\NormalTok{data}\OperatorTok{$}\NormalTok{basic_materials}
\NormalTok{financial =}\StringTok{ }\NormalTok{data}\OperatorTok{$}\NormalTok{financial_services}
\NormalTok{cyclical =}\StringTok{ }\NormalTok{data}\OperatorTok{$}\NormalTok{consumer_cyclical}
\NormalTok{estate =}\StringTok{ }\NormalTok{data}\OperatorTok{$}\NormalTok{real_estate}
\NormalTok{defensive =}\StringTok{ }\NormalTok{data}\OperatorTok{$}\NormalTok{consumer_defensive}
\NormalTok{healthcare =}\StringTok{ }\NormalTok{data}\OperatorTok{$}\NormalTok{healthcare}
\NormalTok{utilities =}\StringTok{ }\NormalTok{data}\OperatorTok{$}\NormalTok{utilities}
\NormalTok{communication =}\StringTok{ }\NormalTok{data}\OperatorTok{$}\NormalTok{communication_services}
\NormalTok{energy =}\StringTok{ }\NormalTok{data}\OperatorTok{$}\NormalTok{energy}
\NormalTok{industrials =}\StringTok{ }\NormalTok{data}\OperatorTok{$}\NormalTok{industrials}
\NormalTok{technology =}\StringTok{ }\NormalTok{data}\OperatorTok{$}\NormalTok{technology}
\NormalTok{labels =}\StringTok{ }\KeywordTok{c}\NormalTok{(}\StringTok{"Basic materials"}\NormalTok{,}
           \StringTok{"Financial services"}\NormalTok{,}
           \StringTok{"Consumer cyclical"}\NormalTok{,}
           \StringTok{"Real estate"}\NormalTok{,}
           \StringTok{"Consumer defensive"}\NormalTok{,}
           \StringTok{"Healthcare"}\NormalTok{,}
           \StringTok{"Utilities"}\NormalTok{,}
           \StringTok{"Communication services"}\NormalTok{,}
           \StringTok{"Energy"}\NormalTok{,}
           \StringTok{"Industrials"}\NormalTok{,}
           \StringTok{"Technology"}\NormalTok{)}
\KeywordTok{boxplot}\NormalTok{(materials,}
\NormalTok{        financial,}
\NormalTok{        cyclical,}
\NormalTok{        estate,}
\NormalTok{        defensive,}
\NormalTok{        healthcare,}
\NormalTok{        utilities,}
\NormalTok{        communication,}
\NormalTok{        energy,}
\NormalTok{        industrials,}
\NormalTok{        technology,}
        \DataTypeTok{names=}\NormalTok{labels,}
        \DataTypeTok{ylab=}\StringTok{"Percentage"}\NormalTok{,}
        \DataTypeTok{las =} \DecValTok{2}\NormalTok{)}
\end{Highlighting}
\end{Shaded}

\includegraphics{projekt_files/figure-latex/unnamed-chunk-4-1.pdf}

\pagebreak

\hypertarget{statistiux10dko-zakljuux10divanje}{%
\section{Statističko
zaključivanje:}\label{statistiux10dko-zakljuux10divanje}}

Kao što je već spomenuto svaki fond ima određeni stil investiranja
(Growth, Blend, Value). Zanima nas razlikuju li se uspješnosti fondova s
obzirom na stil investiranja koji koriste, odnosno želimo provijeriti
imaju li fodnovi s određenim stilom investiranja veće povrate nego
ostali. Za početak želimo vidjeti ravnaju li se povrati svake od tih
kategorija po normalnoj razdiobi kako bi mogli primjeniti analizu
varijance.

\begin{Shaded}
\begin{Highlighting}[]
\NormalTok{data_blend <-}\StringTok{ }\NormalTok{data[data}\OperatorTok{$}\NormalTok{investment }\OperatorTok{==}\StringTok{ }\KeywordTok{c}\NormalTok{(}\StringTok{"Blend"}\NormalTok{),]}
\KeywordTok{hist}\NormalTok{(data_blend}\OperatorTok{$}\NormalTok{fund_mean_annual_return_10years)}
\KeywordTok{qqnorm}\NormalTok{(data_blend}\OperatorTok{$}\NormalTok{fund_mean_annual_return_10years, }\DataTypeTok{main=}\StringTok{"Srednji povrat za zadnjih 10 godina za Blend"}\NormalTok{)}
\KeywordTok{qqline}\NormalTok{(data_blend}\OperatorTok{$}\NormalTok{fund_mean_annual_return_10years, }\DataTypeTok{col=}\StringTok{"blue"}\NormalTok{)}
\end{Highlighting}
\end{Shaded}

\includegraphics[width=0.5\linewidth]{projekt_files/figure-latex/unnamed-chunk-5-1}
\includegraphics[width=0.5\linewidth]{projekt_files/figure-latex/unnamed-chunk-5-2}

\begin{Shaded}
\begin{Highlighting}[]
\NormalTok{data_growth <-}\StringTok{ }\NormalTok{data[data}\OperatorTok{$}\NormalTok{investment }\OperatorTok{==}\StringTok{ }\KeywordTok{c}\NormalTok{(}\StringTok{"Growth"}\NormalTok{),]}
\KeywordTok{hist}\NormalTok{(data_growth}\OperatorTok{$}\NormalTok{fund_mean_annual_return_10years)}
\KeywordTok{qqnorm}\NormalTok{(data_growth}\OperatorTok{$}\NormalTok{fund_mean_annual_return_10years, }\DataTypeTok{main=}\StringTok{"Srednji povrat za zadnjih 10 godina za Growth"}\NormalTok{)}
\KeywordTok{qqline}\NormalTok{(data_growth}\OperatorTok{$}\NormalTok{fund_mean_annual_return_10years, }\DataTypeTok{col=}\StringTok{"blue"}\NormalTok{)}
\end{Highlighting}
\end{Shaded}

\includegraphics[width=0.5\linewidth]{projekt_files/figure-latex/unnamed-chunk-6-1}
\includegraphics[width=0.5\linewidth]{projekt_files/figure-latex/unnamed-chunk-6-2}

\begin{Shaded}
\begin{Highlighting}[]
\NormalTok{data_value <-}\StringTok{ }\NormalTok{data[data}\OperatorTok{$}\NormalTok{investment }\OperatorTok{==}\StringTok{ }\KeywordTok{c}\NormalTok{(}\StringTok{"Value"}\NormalTok{),]}
\KeywordTok{hist}\NormalTok{(data_value}\OperatorTok{$}\NormalTok{fund_mean_annual_return_10years)}
\KeywordTok{qqnorm}\NormalTok{(data_value}\OperatorTok{$}\NormalTok{fund_mean_annual_return_10years, }\DataTypeTok{main=}\StringTok{"Srednji povrat za zadnjih 10 godina za Value"}\NormalTok{)}
\KeywordTok{qqline}\NormalTok{(data_value}\OperatorTok{$}\NormalTok{fund_mean_annual_return_10years, }\DataTypeTok{col=}\StringTok{"blue"}\NormalTok{)}
\end{Highlighting}
\end{Shaded}

\includegraphics[width=0.5\linewidth]{projekt_files/figure-latex/unnamed-chunk-7-1}
\includegraphics[width=0.5\linewidth]{projekt_files/figure-latex/unnamed-chunk-7-2}

Pretpostavke ANOVA testa su: populacije iz grupa međusobno su nezavisne
i normalno distribuirane sa jednakim varijancama. Nezavisnost populacija
teško možemo provjeriti stoga ćemo prepostaviti da su one nezavisne. Iz
priloženih grafova možemo zaključiti da nema većih zakrivljenosti u
podacima te stoga koristimo testove koji bi koristili kao i kada bi se
podaci ravnali po normalnoj distribuciji.

U nastavku vidimo da se varijance populacija razlikuju do na red
veličine te možemo pretpostaviti da su varijance populacija približno
jednake.

\begin{Shaded}
\begin{Highlighting}[]
\NormalTok{data_growth_filtered <-}\StringTok{ }\KeywordTok{na.omit}\NormalTok{(data_growth}\OperatorTok{$}\NormalTok{fund_mean_annual_return_10years)}
\NormalTok{data_blend_filtered <-}\StringTok{ }\KeywordTok{na.omit}\NormalTok{(data_blend}\OperatorTok{$}\NormalTok{fund_mean_annual_return_10years)}
\NormalTok{data_value_filtered <-}\StringTok{ }\KeywordTok{na.omit}\NormalTok{(data_value}\OperatorTok{$}\NormalTok{fund_mean_annual_return_10years)}

\KeywordTok{var}\NormalTok{(data_growth_filtered)}
\end{Highlighting}
\end{Shaded}

\begin{verbatim}
## [1] 0.07227282
\end{verbatim}

\begin{Shaded}
\begin{Highlighting}[]
\KeywordTok{var}\NormalTok{(data_blend_filtered)}
\end{Highlighting}
\end{Shaded}

\begin{verbatim}
## [1] 0.07808235
\end{verbatim}

\begin{Shaded}
\begin{Highlighting}[]
\KeywordTok{var}\NormalTok{(data_value_filtered)}
\end{Highlighting}
\end{Shaded}

\begin{verbatim}
## [1] 0.04714453
\end{verbatim}

\pagebreak

Prvo prikažimo pravokutni dijagram za te tri kategorije.

\begin{Shaded}
\begin{Highlighting}[]
\KeywordTok{boxplot}\NormalTok{(data}\OperatorTok{$}\NormalTok{fund_mean_annual_return_10years[data}\OperatorTok{$}\NormalTok{investment }\OperatorTok{!=}\StringTok{ "<undefined>"}\NormalTok{] }
        \OperatorTok{~}\StringTok{ }\NormalTok{data}\OperatorTok{$}\NormalTok{investment[data}\OperatorTok{$}\NormalTok{investment }\OperatorTok{!=}\StringTok{ "<undefined>"}\NormalTok{],}
        \DataTypeTok{ylab=} \StringTok{"srednji godišnji povrat"}\NormalTok{,}
        \DataTypeTok{xlab=} \StringTok{"Stil investiranja"}\NormalTok{)}
\end{Highlighting}
\end{Shaded}

\includegraphics{projekt_files/figure-latex/unnamed-chunk-9-1.pdf}

Vidimo da se sredine povrata kreću oko sličnih vrijednosti te stoga
vrijedi tesirati potencijalnu jednakost tih sredina. Pretpostavljamo da
su sredine te tri populacije jednake te uz gore navedene pretpostavke
provodimo ANOVA test o jednakosti sredina. Nulta hipoteza je da su
sredine te 3 kategorije jednake, a alternativna hipoteza je da se barem
dvije sredine razlikuju.

\begin{Shaded}
\begin{Highlighting}[]
\NormalTok{res.aov <-}\StringTok{ }\KeywordTok{aov}\NormalTok{(fund_mean_annual_return_10years }\OperatorTok{~}\StringTok{ }\KeywordTok{factor}\NormalTok{(investment), }\DataTypeTok{data =}\NormalTok{ data)}
\KeywordTok{summary}\NormalTok{(res.aov)}
\end{Highlighting}
\end{Shaded}

\begin{verbatim}
##                      Df Sum Sq Mean Sq F value Pr(>F)    
## factor(investment)    3   30.0  10.013   148.4 <2e-16 ***
## Residuals          6379  430.4   0.067                   
## ---
## Signif. codes:  0 '***' 0.001 '**' 0.01 '*' 0.05 '.' 0.1 ' ' 1
## 3 observations deleted due to missingness
\end{verbatim}

Iz rezultata anove definitivno možemo zaključiti da sredine tih uzoraka
nisu jednake te odbacit nultu hipotezu u korist tvrdnje da su sredine
različite. ANOVA nam samo govori da su sredine tih kategorija međusobno
različite, no nas naravno zanima koja od te 3 kategorije ima najveći
srednji povrat. Iz pravokutnog dijagrama se može vidjeti da kategorija
Growth ima nešto veću sredinu nego ostale dvije kategorije te da Blend
ima malo veću sredinu nego Value. Provodimo t-test kako bismo vidjeli
ima li kategorija Growth veću sredinu od kategorije Blend. Pretpostavke
t-testa su normalnost i nezavisnot podataka i one vrijede zbog početnih
pretpostavki ANOVA testa.

Prvo nas zanima jesu li varijance kategorija jednake,a da bismo to
saznali provodimo test o jednakosti varijanci.

\begin{Shaded}
\begin{Highlighting}[]
\KeywordTok{var.test}\NormalTok{(data_growth_filtered, data_blend_filtered)}
\end{Highlighting}
\end{Shaded}

\begin{verbatim}
## 
##  F test to compare two variances
## 
## data:  data_growth_filtered and data_blend_filtered
## F = 0.9256, num df = 2821, denom df = 1802, p-value = 0.06864
## alternative hypothesis: true ratio of variances is not equal to 1
## 95 percent confidence interval:
##  0.8510221 1.0059219
## sample estimates:
## ratio of variances 
##          0.9255974
\end{verbatim}

Vidimo da nam test daje p-vrijednost = 0.0686, što znači da na razini
značajnosti 0.05 ne možemo odbaciti pretpostavku da su varijance
jednake. Budući da smo dobili takvo riješenje, radimo t-test za dvije
populacije koje imaju jednake varijance. Nulta hipoteza t-testa je da su
srednji godišnji povrati tih dviju kategorija jednaki, a alternativna
hipoteza je da populacija Growth ima veći povrat od populacije Blend.

\begin{Shaded}
\begin{Highlighting}[]
\KeywordTok{t.test}\NormalTok{(data_growth}\OperatorTok{$}\NormalTok{fund_mean_annual_return_10years, data_blend}\OperatorTok{$}\NormalTok{fund_mean_annual_return_10years, }\DataTypeTok{alt =} \StringTok{"greater"}\NormalTok{, }\DataTypeTok{var.equal =} \OtherTok{TRUE}\NormalTok{)}
\end{Highlighting}
\end{Shaded}

\begin{verbatim}
## 
##  Two Sample t-test
## 
## data:  data_growth$fund_mean_annual_return_10years and data_blend$fund_mean_annual_return_10years
## t = 13.387, df = 4623, p-value < 2.2e-16
## alternative hypothesis: true difference in means is greater than 0
## 95 percent confidence interval:
##  0.09664701        Inf
## sample estimates:
## mean of x mean of y 
##  1.217055  1.106866
\end{verbatim}

Vidimo da je p-vrijednost provedenog testa izuzetno mala, što nam govori
u prilog odbacivanja nulte hipoteze o jednakosti sredina. Možemo
zaključiti da kategorija Growth ima veći povrat od kategorije Blend.

\hypertarget{utjecaj-stila-investiranja-na-pe-ratio}{%
\subsection{Utjecaj stila investiranja na P/E
ratio}\label{utjecaj-stila-investiranja-na-pe-ratio}}

Zanima nas utječe li stil investiranja na Price-to-Earnings ratio.
Price-to-Earnings ratio je omjer trenutačne tržišne cijene i zarade po
dionici tijekom protekle godine. Budući da stil investiranja Growth
obilježava ulaganje u kompanije za koje se očekuje da će brže rasti, a
stil investiranja Value ulaganje trenutno podcijenjene kompanije, za
očekivati je da se njihov Price-to-Earnings ratio razlikuje.

Iz priloženih grafova vidimo da se podaci za Blend stil investiranja
bitno razlikuju od normalne razdiobe te nam to sugerira da je bolje
napraviti neparametarsku alternativu ANOVA testa, a to je Kruskal-Wallis
test.

\begin{Shaded}
\begin{Highlighting}[]
\NormalTok{data_blend <-}\StringTok{ }\NormalTok{data[data}\OperatorTok{$}\NormalTok{investment }\OperatorTok{==}\StringTok{ }\KeywordTok{c}\NormalTok{(}\StringTok{"Blend"}\NormalTok{),]}
\KeywordTok{hist}\NormalTok{(data_blend}\OperatorTok{$}\NormalTok{price_earnings)}
\KeywordTok{qqnorm}\NormalTok{(data_blend}\OperatorTok{$}\NormalTok{price_earnings, }\DataTypeTok{main=}\StringTok{"p/e ratio"}\NormalTok{)}
\KeywordTok{qqline}\NormalTok{(data_blend}\OperatorTok{$}\NormalTok{price_earnings, }\DataTypeTok{col=}\StringTok{"blue"}\NormalTok{)}
\end{Highlighting}
\end{Shaded}

\includegraphics[width=0.5\linewidth]{projekt_files/figure-latex/unnamed-chunk-13-1}
\includegraphics[width=0.5\linewidth]{projekt_files/figure-latex/unnamed-chunk-13-2}

\begin{Shaded}
\begin{Highlighting}[]
\NormalTok{data_growth <-}\StringTok{ }\NormalTok{data[data}\OperatorTok{$}\NormalTok{investment }\OperatorTok{==}\StringTok{ }\KeywordTok{c}\NormalTok{(}\StringTok{"Growth"}\NormalTok{),]}
\KeywordTok{hist}\NormalTok{(data_growth}\OperatorTok{$}\NormalTok{price_earnings)}
\KeywordTok{qqnorm}\NormalTok{(data_growth}\OperatorTok{$}\NormalTok{price_earnings, }\DataTypeTok{main=}\StringTok{"p/e ratio"}\NormalTok{)}
\KeywordTok{qqline}\NormalTok{(data_growth}\OperatorTok{$}\NormalTok{price_earnings, }\DataTypeTok{col=}\StringTok{"blue"}\NormalTok{)}
\end{Highlighting}
\end{Shaded}

\includegraphics[width=0.5\linewidth]{projekt_files/figure-latex/unnamed-chunk-14-1}
\includegraphics[width=0.5\linewidth]{projekt_files/figure-latex/unnamed-chunk-14-2}

\begin{Shaded}
\begin{Highlighting}[]
\NormalTok{data_value <-}\StringTok{ }\NormalTok{data[data}\OperatorTok{$}\NormalTok{investment }\OperatorTok{==}\StringTok{ }\KeywordTok{c}\NormalTok{(}\StringTok{"Value"}\NormalTok{),]}
\KeywordTok{hist}\NormalTok{(data_value}\OperatorTok{$}\NormalTok{price_earnings)}
\KeywordTok{qqnorm}\NormalTok{(data_value}\OperatorTok{$}\NormalTok{price_earnings, }\DataTypeTok{main=}\StringTok{"p/e ratio"}\NormalTok{)}
\KeywordTok{qqline}\NormalTok{(data_value}\OperatorTok{$}\NormalTok{price_earnings, }\DataTypeTok{col=}\StringTok{"blue"}\NormalTok{)}
\end{Highlighting}
\end{Shaded}

\includegraphics[width=0.5\linewidth]{projekt_files/figure-latex/unnamed-chunk-15-1}
\includegraphics[width=0.5\linewidth]{projekt_files/figure-latex/unnamed-chunk-15-2}

Uz pretpostavku jednakosti distribucija do na translaciju,
Kruskal-Wallis test se može interpretirati kao test jednakosti sredina.
Nulta hipoteza je da su sredine svih kategorija jednake, a alternativna
hipoteza da se barem dvije sredine razlikuju.

\begin{Shaded}
\begin{Highlighting}[]
\KeywordTok{boxplot}\NormalTok{(data}\OperatorTok{$}\NormalTok{price_earnings[data}\OperatorTok{$}\NormalTok{investment }\OperatorTok{!=}\StringTok{ "<undefined>"}\NormalTok{] }\OperatorTok{~}\StringTok{ }\NormalTok{data}\OperatorTok{$}\NormalTok{investment[data}\OperatorTok{$}\NormalTok{investment }\OperatorTok{!=}\StringTok{ "<undefined>"}\NormalTok{],}
        \DataTypeTok{ylab=} \StringTok{"p/e ratio"}\NormalTok{,}
        \DataTypeTok{xlab=} \StringTok{"Stil investiranja"}\NormalTok{)}
\end{Highlighting}
\end{Shaded}

\includegraphics{projekt_files/figure-latex/unnamed-chunk-16-1.pdf}

\begin{Shaded}
\begin{Highlighting}[]
\KeywordTok{kruskal.test}\NormalTok{(data}\OperatorTok{$}\NormalTok{price_earnings }\OperatorTok{~}\StringTok{ }\NormalTok{data}\OperatorTok{$}\NormalTok{investment, }\DataTypeTok{data =}\NormalTok{ data)}
\end{Highlighting}
\end{Shaded}

\begin{verbatim}
## 
##  Kruskal-Wallis rank sum test
## 
## data:  data$price_earnings by data$investment
## Kruskal-Wallis chi-squared = 3253.2, df = 3, p-value < 2.2e-16
\end{verbatim}

Možemo vidjeti da nam Kruskal-Wallis ovdje sugerira da možemo odbaciti
nultu hipotezu da su sredine jednake jer je p-vrijednost izuzetno mala.
Sada kada znamo da nemaju sve kategorije jednak srednji P/E ratio, može
nas zanimati koja koategorija ima najmanji P/E ratio. Zanima nas imaju
li fondovi sa stilom investiranja Value manji P/E ratio nego oni sa
Blend stilom investiranja. Budući da se razdiobe podataka razlikuju od
normalne, koristit ćemo neparametarsku alternativu T-testa,
Mann-Whitney-Willcoxonov test. Mann-Whitney-Willcoxonov test se uz
pretpostavku da su uzorci iz istih distribucija, svodi na test
jednakosti sredina. Nulta hipoteza nam je da su sredine jednake, a
alternativa da se razlikuju.

\begin{Shaded}
\begin{Highlighting}[]
\NormalTok{res <-}\StringTok{ }\KeywordTok{wilcox.test}\NormalTok{(data_blend}\OperatorTok{$}\NormalTok{price_earnings, data_value}\OperatorTok{$}\NormalTok{price_earnings)}
\NormalTok{res}
\end{Highlighting}
\end{Shaded}

\begin{verbatim}
## 
##  Wilcoxon rank sum test with continuity correction
## 
## data:  data_blend$price_earnings and data_value$price_earnings
## W = 2285158, p-value < 2.2e-16
## alternative hypothesis: true location shift is not equal to 0
\end{verbatim}

Mann-Whitney-Willcoxonov test daje nam izuzetno malo vrijednost iz čega
možemo zaključiti da se srednji P/E ratio za dvije promatrane grupe
razlikuje, odnosno da kategorija Value ima manji P/E ratio od kategorije
Blend.

\hypertarget{uspjeux161nost-fonda-s-obzirom-na-kategoriju}{%
\subsection{Uspješnost fonda s obzirom na
kategoriju}\label{uspjeux161nost-fonda-s-obzirom-na-kategoriju}}

Zanimljivo je promatrati razlikuje li se uspješnost fonda s obzirom na
kategoriju. Budući da u našim podatcima ima oko 50 različitih
kategorija, nema smisla provoditi anovu. Najzanimljivije bi bilo
promatrati one kategorije kojima najviše fondova pripada jer one
vjerojatno prikazuju najrealniju sliku. Najviše fondova pripada u
kategorije Large Blend, Large Value, Large Growth, Small Blend, Small
Value, Small Growth koje većinom imaju istoimeni stil; Blend, Growth,
Value. Utjecaj stila investiranja na uspješnost fonda smo već proučavali
pa ćemo ovdje uzeti neke druge kategorije koje se čine zanimljive i lako
razumljive, a pripada im puno fondova.To su World Large Stock,
Technology i Real Estate.

\begin{Shaded}
\begin{Highlighting}[]
\CommentTok{#data_cat <- data$category}
\CommentTok{#as.data.frame(table(data_cat))}

\NormalTok{filtered_by_categories <-}\StringTok{ }\NormalTok{data[data}\OperatorTok{$}\NormalTok{category }\OperatorTok{==}\StringTok{ "World Large Stock"} \OperatorTok{|}\StringTok{ }\NormalTok{data}\OperatorTok{$}\NormalTok{category }\OperatorTok{==}\StringTok{ "Technology"} \OperatorTok{|}\StringTok{ }\NormalTok{data}\OperatorTok{$}\NormalTok{category }\OperatorTok{==}\StringTok{ "Real Estate"}\NormalTok{,]}

\KeywordTok{boxplot}\NormalTok{(filtered_by_categories}\OperatorTok{$}\NormalTok{fund_mean_annual_return_10years }\OperatorTok{~}\StringTok{ }\NormalTok{filtered_by_categories}\OperatorTok{$}\NormalTok{category,}
        \DataTypeTok{ylab =} \StringTok{"srednji godišnji povrat"}\NormalTok{,}
        \DataTypeTok{xlab =} \StringTok{"kategorije"}\NormalTok{)}
\end{Highlighting}
\end{Shaded}

\includegraphics{projekt_files/figure-latex/unnamed-chunk-18-1.pdf}

\begin{Shaded}
\begin{Highlighting}[]
\NormalTok{data_re <-}\StringTok{ }\NormalTok{data[data}\OperatorTok{$}\NormalTok{category }\OperatorTok{==}\StringTok{ }\KeywordTok{c}\NormalTok{(}\StringTok{"Real Estate"}\NormalTok{),]}
\KeywordTok{hist}\NormalTok{(data_re}\OperatorTok{$}\NormalTok{fund_mean_annual_return_10years)}
\end{Highlighting}
\end{Shaded}

\includegraphics{projekt_files/figure-latex/unnamed-chunk-19-1.pdf}

\begin{Shaded}
\begin{Highlighting}[]
\KeywordTok{qqnorm}\NormalTok{(data_re}\OperatorTok{$}\NormalTok{fund_mean_annual_return_10years, }\DataTypeTok{main=}\StringTok{"srednji godišnji povrat 10 godina"}\NormalTok{)}
\KeywordTok{qqline}\NormalTok{(data_re}\OperatorTok{$}\NormalTok{fund_mean_annual_return_10years, }\DataTypeTok{col=}\StringTok{"blue"}\NormalTok{)}
\end{Highlighting}
\end{Shaded}

\includegraphics{projekt_files/figure-latex/unnamed-chunk-19-2.pdf}

\begin{Shaded}
\begin{Highlighting}[]
\NormalTok{data_tech <-}\StringTok{ }\NormalTok{data[data}\OperatorTok{$}\NormalTok{category }\OperatorTok{==}\StringTok{ }\KeywordTok{c}\NormalTok{(}\StringTok{"Technology"}\NormalTok{),]}
\KeywordTok{hist}\NormalTok{(data_tech}\OperatorTok{$}\NormalTok{fund_mean_annual_return_10years)}
\end{Highlighting}
\end{Shaded}

\includegraphics{projekt_files/figure-latex/unnamed-chunk-19-3.pdf}

\begin{Shaded}
\begin{Highlighting}[]
\KeywordTok{qqnorm}\NormalTok{(data_tech}\OperatorTok{$}\NormalTok{fund_mean_annual_return_10years, }\DataTypeTok{main=}\StringTok{"srednji godišnji povrat 10 godina"}\NormalTok{)}
\KeywordTok{qqline}\NormalTok{(data_tech}\OperatorTok{$}\NormalTok{fund_mean_annual_return_10years, }\DataTypeTok{col=}\StringTok{"blue"}\NormalTok{)}
\end{Highlighting}
\end{Shaded}

\includegraphics{projekt_files/figure-latex/unnamed-chunk-19-4.pdf}

\begin{Shaded}
\begin{Highlighting}[]
\NormalTok{data_wls <-}\StringTok{ }\NormalTok{data[data}\OperatorTok{$}\NormalTok{category }\OperatorTok{==}\StringTok{ }\KeywordTok{c}\NormalTok{(}\StringTok{"World Large Stock"}\NormalTok{),]}
\KeywordTok{hist}\NormalTok{(data_wls}\OperatorTok{$}\NormalTok{fund_mean_annual_return_10years)}
\end{Highlighting}
\end{Shaded}

\includegraphics{projekt_files/figure-latex/unnamed-chunk-19-5.pdf}

\begin{Shaded}
\begin{Highlighting}[]
\KeywordTok{qqnorm}\NormalTok{(data_wls}\OperatorTok{$}\NormalTok{fund_mean_annual_return_10years, }\DataTypeTok{main=}\StringTok{"srednji godišnji povrat 10 godina"}\NormalTok{)}
\KeywordTok{qqline}\NormalTok{(data_wls}\OperatorTok{$}\NormalTok{fund_mean_annual_return_10years, }\DataTypeTok{col=}\StringTok{"blue"}\NormalTok{)}
\end{Highlighting}
\end{Shaded}

\includegraphics{projekt_files/figure-latex/unnamed-chunk-19-6.pdf}

Iz priloženih grafova možemo pretpostaviti da podatci ne odskaču
drastično od normalne distribucije. Uz pretpostavku nezavisnosti
podataka, normalnosti te jednakosti varijanci, možemo provesti ANOVA
test. Nulta hipoteza je da su srednji godišnji povrati jednaki u sve tri
kategorije, a alternativna da se barem dva razlikuju.

\begin{Shaded}
\begin{Highlighting}[]
\NormalTok{data_re_filtered <-}\StringTok{ }\KeywordTok{na.omit}\NormalTok{(data_re}\OperatorTok{$}\NormalTok{fund_mean_annual_return_10years)}
\NormalTok{data_tech_filtered <-}\StringTok{ }\KeywordTok{na.omit}\NormalTok{(data_tech}\OperatorTok{$}\NormalTok{fund_mean_annual_return_10years)}
\NormalTok{data_wls_filtered <-}\StringTok{ }\KeywordTok{na.omit}\NormalTok{(data_wls}\OperatorTok{$}\NormalTok{fund_mean_annual_return_10years)}

\KeywordTok{var}\NormalTok{(data_re_filtered)}
\end{Highlighting}
\end{Shaded}

\begin{verbatim}
## [1] 0.02335224
\end{verbatim}

\begin{Shaded}
\begin{Highlighting}[]
\KeywordTok{var}\NormalTok{(data_tech_filtered)}
\end{Highlighting}
\end{Shaded}

\begin{verbatim}
## [1] 0.09662187
\end{verbatim}

\begin{Shaded}
\begin{Highlighting}[]
\KeywordTok{var}\NormalTok{(data_wls_filtered)}
\end{Highlighting}
\end{Shaded}

\begin{verbatim}
## [1] 0.02697088
\end{verbatim}

\begin{Shaded}
\begin{Highlighting}[]
\NormalTok{res.aov <-}\StringTok{ }\KeywordTok{aov}\NormalTok{(fund_mean_annual_return_10years }\OperatorTok{~}\StringTok{ }\KeywordTok{factor}\NormalTok{(category), }\DataTypeTok{data =}\NormalTok{ filtered_by_categories)}
\KeywordTok{summary}\NormalTok{(res.aov)}
\end{Highlighting}
\end{Shaded}

\begin{verbatim}
##                   Df Sum Sq Mean Sq F value Pr(>F)    
## factor(category)   2  30.13  15.065   393.2 <2e-16 ***
## Residuals        585  22.41   0.038                   
## ---
## Signif. codes:  0 '***' 0.001 '**' 0.01 '*' 0.05 '.' 0.1 ' ' 1
\end{verbatim}

ANOVA nam govori u prilog alternativne hipoteze, a to je da su sredine
barem dvije of te tri kategorije različite. Iz pravokutnog dijagrama se
vidi da su sredine kategorija Real Estate i Technology iznad sredine od
World Large Stock, stoga ćemo uz pomoć t-testa provjeriti imaju li Real
Estate i Technology jednake srednje povrate. Iako smo anovu radili pod
pretpostavkom da su varijance jednake za t-test ćemo testirat jesu li
varijance za dvije promatrane skupine jednake kako bi rezultat bio
pouzdaniji.

\begin{Shaded}
\begin{Highlighting}[]
\KeywordTok{var.test}\NormalTok{(data_re_filtered, data_tech_filtered)}
\end{Highlighting}
\end{Shaded}

\begin{verbatim}
## 
##  F test to compare two variances
## 
## data:  data_re_filtered and data_tech_filtered
## F = 0.24169, num df = 149, denom df = 103, p-value = 4.167e-15
## alternative hypothesis: true ratio of variances is not equal to 1
## 95 percent confidence interval:
##  0.1681521 0.3433157
## sample estimates:
## ratio of variances 
##          0.2416869
\end{verbatim}

Test o jednakosti varijanci odbacuje da su varijance jednake stoga
provodimo t-test za nezavisne uzorke za populacije s različitim
varijancama.

\begin{Shaded}
\begin{Highlighting}[]
\KeywordTok{t.test}\NormalTok{(data_tech}\OperatorTok{$}\NormalTok{fund_mean_annual_return_10years, data_re}\OperatorTok{$}\NormalTok{fund_mean_annual_return_10years, }\DataTypeTok{alt =} \StringTok{"greater"}\NormalTok{, }\DataTypeTok{var.equal =} \OtherTok{FALSE}\NormalTok{)}
\end{Highlighting}
\end{Shaded}

\begin{verbatim}
## 
##  Welch Two Sample t-test
## 
## data:  data_tech$fund_mean_annual_return_10years and data_re$fund_mean_annual_return_10years
## t = 1.3054, df = 137.74, p-value = 0.09697
## alternative hypothesis: true difference in means is greater than 0
## 95 percent confidence interval:
##  -0.0115471        Inf
## sample estimates:
## mean of x mean of y 
##  1.519327  1.476333
\end{verbatim}

T-test nam vraća p-vrijednost = 0,09697 što znači da ne možemo na razini
pouzdanosti 0.05 odbaciti nultu hipotezu da su sredine te dvije skupine
jednake.Ne možemo zaključiti da neka od ovih dvije kategorija garantira
prosječno veći povrat.

\hypertarget{utjecaj-tehnologije-na-uspjeux161nosti-fonda}{%
\subsection{Utjecaj tehnologije na uspješnosti
fonda}\label{utjecaj-tehnologije-na-uspjeux161nosti-fonda}}

Testira se postoji li linearna zavisnost između udjela ulaganja u
tehnologiju i uspješnosti fonda.

H0: fondovi imaju jednaku uspješnost bez obzira na udio koji ulažu u
tehnologiju H1: fondovi imaju veću uspješnost ako imaju veći udio
ulaganja u tehnologiju

\begin{Shaded}
\begin{Highlighting}[]
\NormalTok{uspjesnost =}\StringTok{ }\NormalTok{data}\OperatorTok{$}\NormalTok{fund_mean_annual_return_10years}
\NormalTok{udio_tech =}\StringTok{ }\NormalTok{data}\OperatorTok{$}\NormalTok{technology}
\NormalTok{model =}\StringTok{ }\KeywordTok{lm}\NormalTok{(uspjesnost}\OperatorTok{~}\NormalTok{udio_tech)}
\KeywordTok{plot}\NormalTok{(udio_tech, uspjesnost)}
\KeywordTok{abline}\NormalTok{(model, }\DataTypeTok{col =} \StringTok{"red"}\NormalTok{)}
\end{Highlighting}
\end{Shaded}

\includegraphics{projekt_files/figure-latex/unnamed-chunk-23-1.pdf}

\begin{Shaded}
\begin{Highlighting}[]
\KeywordTok{summary}\NormalTok{(model)}
\end{Highlighting}
\end{Shaded}

\begin{verbatim}
## 
## Call:
## lm(formula = uspjesnost ~ udio_tech)
## 
## Residuals:
##      Min       1Q   Median       3Q      Max 
## -1.77966 -0.11453  0.04423  0.14796  0.76034 
## 
## Coefficients:
##              Estimate Std. Error t value Pr(>|t|)    
## (Intercept) 0.9996566  0.0054782   182.5   <2e-16 ***
## udio_tech   0.0078387  0.0002382    32.9   <2e-16 ***
## ---
## Signif. codes:  0 '***' 0.001 '**' 0.01 '*' 0.05 '.' 0.1 ' ' 1
## 
## Residual standard error: 0.2484 on 6381 degrees of freedom
##   (3 observations deleted due to missingness)
## Multiple R-squared:  0.1451, Adjusted R-squared:  0.1449 
## F-statistic:  1083 on 1 and 6381 DF,  p-value: < 2.2e-16
\end{verbatim}

Postoji indicija da je zavisnost prisutna, ali potrebno je dodatno
provjeriti normalnost te napraviti analizu reziduala.

\begin{Shaded}
\begin{Highlighting}[]
\KeywordTok{qqnorm}\NormalTok{(}\KeywordTok{rstandard}\NormalTok{(model))}
\KeywordTok{qqline}\NormalTok{(}\KeywordTok{rstandard}\NormalTok{(model))}
\end{Highlighting}
\end{Shaded}

\includegraphics{projekt_files/figure-latex/unnamed-chunk-24-1.pdf}

\begin{Shaded}
\begin{Highlighting}[]
\KeywordTok{plot}\NormalTok{(}\KeywordTok{fitted}\NormalTok{(model), }\KeywordTok{resid}\NormalTok{(model))}
\KeywordTok{abline}\NormalTok{(}\DecValTok{0}\NormalTok{,}\DecValTok{0}\NormalTok{, }\DataTypeTok{col=}\StringTok{"red"}\NormalTok{)}
\end{Highlighting}
\end{Shaded}

\includegraphics{projekt_files/figure-latex/unnamed-chunk-24-2.pdf}

\begin{Shaded}
\begin{Highlighting}[]
\KeywordTok{c}\NormalTok{(}\StringTok{"Pearson: "}\NormalTok{, }\KeywordTok{cor}\NormalTok{(udio_tech, uspjesnost, }\DataTypeTok{method =} \StringTok{"pearson"}\NormalTok{, }\DataTypeTok{use =} \StringTok{"complete.obs"}\NormalTok{))}
\end{Highlighting}
\end{Shaded}

\begin{verbatim}
## [1] "Pearson: "         "0.380871018407125"
\end{verbatim}

Iako QQ-plot ukazuje na to da su podaci iz normalne distribucije te
reziduali ne pokazuju neki očit uzorak, Pearsonov koeficijent koji je
dodatno izračunat ne ukazuje na visok stupanj zavisnosti između udjela
ulaganja u tehnologiju i uspješnosti fonda.

Ispituje se postoji li linearna zavisnost između cijene i uspješnosti
fonda.

H0: ne postoji linearna zavisnost između cijene i uspješnosti fonda H1:
postoji linearna zavisnost cijene i uspješnosti fonda

\begin{Shaded}
\begin{Highlighting}[]
\NormalTok{uspjesnost =}\StringTok{ }\NormalTok{data}\OperatorTok{$}\NormalTok{fund_mean_annual_return_10years}
\NormalTok{cijena =}\StringTok{ }\NormalTok{data}\OperatorTok{$}\NormalTok{net_annual_expense_ratio_fund}
\NormalTok{model =}\StringTok{ }\KeywordTok{lm}\NormalTok{(uspjesnost}\OperatorTok{~}\NormalTok{cijena)}
\KeywordTok{plot}\NormalTok{(cijena, uspjesnost)}
\KeywordTok{abline}\NormalTok{(model, }\DataTypeTok{col=}\StringTok{"red"}\NormalTok{)}
\end{Highlighting}
\end{Shaded}

\includegraphics{projekt_files/figure-latex/unnamed-chunk-25-1.pdf}

\begin{Shaded}
\begin{Highlighting}[]
\KeywordTok{summary}\NormalTok{(model)}
\end{Highlighting}
\end{Shaded}

\begin{verbatim}
## 
## Call:
## lm(formula = uspjesnost ~ cijena)
## 
## Residuals:
##      Min       1Q   Median       3Q      Max 
## -1.62686 -0.12138  0.03087  0.15789  0.92020 
## 
## Coefficients:
##              Estimate Std. Error t value Pr(>|t|)    
## (Intercept)  1.323824   0.007813  169.43   <2e-16 ***
## cijena      -0.149052   0.006041  -24.68   <2e-16 ***
## ---
## Signif. codes:  0 '***' 0.001 '**' 0.01 '*' 0.05 '.' 0.1 ' ' 1
## 
## Residual standard error: 0.2567 on 6381 degrees of freedom
##   (3 observations deleted due to missingness)
## Multiple R-squared:  0.08711,    Adjusted R-squared:  0.08696 
## F-statistic: 608.9 on 1 and 6381 DF,  p-value: < 2.2e-16
\end{verbatim}

Potrebno je provjeriti normalnost podataka te napraviti analizu
reziduala.

\begin{Shaded}
\begin{Highlighting}[]
\KeywordTok{qqnorm}\NormalTok{(}\KeywordTok{rstandard}\NormalTok{(model))}
\KeywordTok{qqline}\NormalTok{(}\KeywordTok{rstandard}\NormalTok{(model))}
\end{Highlighting}
\end{Shaded}

\includegraphics{projekt_files/figure-latex/unnamed-chunk-26-1.pdf}

\begin{Shaded}
\begin{Highlighting}[]
\KeywordTok{plot}\NormalTok{(}\KeywordTok{fitted}\NormalTok{(model), }\KeywordTok{resid}\NormalTok{(model))}
\KeywordTok{abline}\NormalTok{(}\DecValTok{0}\NormalTok{,}\DecValTok{0}\NormalTok{, }\DataTypeTok{col=}\StringTok{"red"}\NormalTok{)}
\end{Highlighting}
\end{Shaded}

\includegraphics{projekt_files/figure-latex/unnamed-chunk-26-2.pdf}
Analiza reziduala pokazuje visoku grupiranost te se podaci ne mogu
koristiti za linearnu regresiju.

H0: razlika je jednaka za skuplje i jeftinije fondove H1: razlika je
manja za skuplje fondove

\begin{Shaded}
\begin{Highlighting}[]
\NormalTok{data1 =}\StringTok{ }\KeywordTok{data.frame}\NormalTok{(data}\OperatorTok{$}\NormalTok{fund_mean_annual_return_10years, data}\OperatorTok{$}\NormalTok{fund_mean_annual_return_5years, data}\OperatorTok{$}\NormalTok{fund_mean_annual_return_3years)}
\NormalTok{data2 =}\StringTok{ }\KeywordTok{data.frame}\NormalTok{(data}\OperatorTok{$}\NormalTok{net_annual_expense_ratio_fund)}
\NormalTok{data2}\OperatorTok{$}\NormalTok{Min <-}\KeywordTok{apply}\NormalTok{(data1,}\DecValTok{1}\NormalTok{,}\DataTypeTok{FUN=}\NormalTok{min)}
\NormalTok{data2}\OperatorTok{$}\NormalTok{Max <-}\KeywordTok{apply}\NormalTok{(data1,}\DecValTok{1}\NormalTok{,}\DataTypeTok{FUN=}\NormalTok{max)}
\NormalTok{diff =}\StringTok{ }\NormalTok{data2}\OperatorTok{$}\NormalTok{Max }\OperatorTok{-}\StringTok{ }\NormalTok{data2}\OperatorTok{$}\NormalTok{Min}
\NormalTok{model =}\StringTok{ }\KeywordTok{lm}\NormalTok{(diff }\OperatorTok{~}\StringTok{ }\NormalTok{data2}\OperatorTok{$}\NormalTok{data.net_annual_expense_ratio_fund)}
\KeywordTok{plot}\NormalTok{(data2}\OperatorTok{$}\NormalTok{data.net_annual_expense_ratio_fund, diff)}
\KeywordTok{abline}\NormalTok{(model, }\DataTypeTok{col =} \StringTok{"red"}\NormalTok{)}
\end{Highlighting}
\end{Shaded}

\includegraphics{projekt_files/figure-latex/unnamed-chunk-27-1.pdf}

\begin{Shaded}
\begin{Highlighting}[]
\KeywordTok{qqnorm}\NormalTok{(}\KeywordTok{rstandard}\NormalTok{(model))}
\KeywordTok{qqline}\NormalTok{(}\KeywordTok{rstandard}\NormalTok{(model))}
\end{Highlighting}
\end{Shaded}

\includegraphics{projekt_files/figure-latex/unnamed-chunk-27-2.pdf}

\begin{Shaded}
\begin{Highlighting}[]
\KeywordTok{plot}\NormalTok{(}\KeywordTok{fitted}\NormalTok{(model), }\KeywordTok{resid}\NormalTok{(model))}
\KeywordTok{abline}\NormalTok{(}\DecValTok{0}\NormalTok{,}\DecValTok{0}\NormalTok{)}
\end{Highlighting}
\end{Shaded}

\includegraphics{projekt_files/figure-latex/unnamed-chunk-27-3.pdf}

\begin{Shaded}
\begin{Highlighting}[]
\KeywordTok{summary}\NormalTok{(model)}
\end{Highlighting}
\end{Shaded}

\begin{verbatim}
## 
## Call:
## lm(formula = diff ~ data2$data.net_annual_expense_ratio_fund)
## 
## Residuals:
##      Min       1Q   Median       3Q      Max 
## -0.60833 -0.13780 -0.02247  0.11160  1.57705 
## 
## Coefficients:
##                                          Estimate Std. Error t value Pr(>|t|)
## (Intercept)                              0.514684   0.005893  87.338   <2e-16
## data2$data.net_annual_expense_ratio_fund 0.042665   0.004556   9.365   <2e-16
##                                             
## (Intercept)                              ***
## data2$data.net_annual_expense_ratio_fund ***
## ---
## Signif. codes:  0 '***' 0.001 '**' 0.01 '*' 0.05 '.' 0.1 ' ' 1
## 
## Residual standard error: 0.1936 on 6381 degrees of freedom
##   (3 observations deleted due to missingness)
## Multiple R-squared:  0.01356,    Adjusted R-squared:  0.0134 
## F-statistic:  87.7 on 1 and 6381 DF,  p-value: < 2.2e-16
\end{verbatim}

\begin{Shaded}
\begin{Highlighting}[]
\KeywordTok{qqplot}\NormalTok{(data2}\OperatorTok{$}\NormalTok{data.net_annual_expense_ratio_fund, diff)}
\end{Highlighting}
\end{Shaded}

\includegraphics{projekt_files/figure-latex/unnamed-chunk-27-4.pdf}

H0: fondovi imaju jednaku uspješnost bez obzira na udio koji ulažu u
tehnologiju H1: fondovi imaju veću uspješnost ako imaju veći udio
ulaganja u tehnologiju

\begin{Shaded}
\begin{Highlighting}[]
\NormalTok{data1 =}\StringTok{ }\KeywordTok{subset}\NormalTok{(data, technology }\OperatorTok{>}\StringTok{ }\DecValTok{0}\NormalTok{)}
\NormalTok{uspjesnost =}\StringTok{ }\NormalTok{data1}\OperatorTok{$}\NormalTok{fund_mean_annual_return_10years}
\NormalTok{udio_tech =}\StringTok{ }\KeywordTok{sqrt}\NormalTok{(data1}\OperatorTok{$}\NormalTok{technology)}
\NormalTok{model =}\StringTok{ }\KeywordTok{lm}\NormalTok{(uspjesnost}\OperatorTok{~}\NormalTok{udio_tech)}
\KeywordTok{plot}\NormalTok{(udio_tech, uspjesnost)}
\KeywordTok{abline}\NormalTok{(model, }\DataTypeTok{col =} \StringTok{"red"}\NormalTok{)}
\end{Highlighting}
\end{Shaded}

\includegraphics{projekt_files/figure-latex/unnamed-chunk-28-1.pdf}

\begin{Shaded}
\begin{Highlighting}[]
\KeywordTok{qqnorm}\NormalTok{(}\KeywordTok{rstandard}\NormalTok{(model))}
\KeywordTok{qqline}\NormalTok{(}\KeywordTok{rstandard}\NormalTok{(model))}
\end{Highlighting}
\end{Shaded}

\includegraphics{projekt_files/figure-latex/unnamed-chunk-28-2.pdf}

\begin{Shaded}
\begin{Highlighting}[]
\KeywordTok{summary}\NormalTok{(model)}
\end{Highlighting}
\end{Shaded}

\begin{verbatim}
## 
## Call:
## lm(formula = uspjesnost ~ udio_tech)
## 
## Residuals:
##      Min       1Q   Median       3Q      Max 
## -1.29455 -0.11198  0.03838  0.14063  0.83113 
## 
## Coefficients:
##             Estimate Std. Error t value Pr(>|t|)    
## (Intercept) 0.883469   0.009609   91.94   <2e-16 ***
## udio_tech   0.064626   0.002149   30.07   <2e-16 ***
## ---
## Signif. codes:  0 '***' 0.001 '**' 0.01 '*' 0.05 '.' 0.1 ' ' 1
## 
## Residual standard error: 0.2305 on 6043 degrees of freedom
##   (3 observations deleted due to missingness)
## Multiple R-squared:  0.1302, Adjusted R-squared:   0.13 
## F-statistic: 904.3 on 1 and 6043 DF,  p-value: < 2.2e-16
\end{verbatim}

Provjeravamo linearnu zavisnost veličine imovine upravljanja fonda i
povrata.

\begin{Shaded}
\begin{Highlighting}[]
\NormalTok{fondovi.data =}\StringTok{ }\NormalTok{data[,}\KeywordTok{c}\NormalTok{(}\StringTok{"net_assets"}\NormalTok{,}\StringTok{"fund_mean_annual_return_10years"}\NormalTok{)]}
\KeywordTok{colnames}\NormalTok{(fondovi.data) =}\StringTok{ }\KeywordTok{c}\NormalTok{(}\StringTok{"velicina"}\NormalTok{, }\StringTok{"povrat"}\NormalTok{)}
\NormalTok{fondovi.data =}\StringTok{ }\KeywordTok{na.omit}\NormalTok{(fondovi.data)}

\NormalTok{log_velicina =}\StringTok{ }\KeywordTok{log}\NormalTok{(fondovi.data}\OperatorTok{$}\NormalTok{velicina)}

\KeywordTok{plot}\NormalTok{(}\KeywordTok{log}\NormalTok{(fondovi.data}\OperatorTok{$}\NormalTok{velicina), fondovi.data}\OperatorTok{$}\NormalTok{povrat)}

\NormalTok{fit.velicine =}\StringTok{ }\KeywordTok{lm}\NormalTok{(fondovi.data}\OperatorTok{$}\NormalTok{povrat}\OperatorTok{~}\NormalTok{log_velicina)}

\KeywordTok{abline}\NormalTok{(fit.velicine)}
\end{Highlighting}
\end{Shaded}

\includegraphics{projekt_files/figure-latex/unnamed-chunk-29-1.pdf}

\begin{Shaded}
\begin{Highlighting}[]
\KeywordTok{summary}\NormalTok{(fit.velicine)}
\end{Highlighting}
\end{Shaded}

\begin{verbatim}
## 
## Call:
## lm(formula = fondovi.data$povrat ~ log_velicina)
## 
## Residuals:
##      Min       1Q   Median       3Q      Max 
## -1.83026 -0.12369  0.04054  0.16779  0.74719 
## 
## Coefficients:
##              Estimate Std. Error t value Pr(>|t|)    
## (Intercept)  0.352773   0.036848   9.574   <2e-16 ***
## log_velicina 0.038969   0.001798  21.670   <2e-16 ***
## ---
## Signif. codes:  0 '***' 0.001 '**' 0.01 '*' 0.05 '.' 0.1 ' ' 1
## 
## Residual standard error: 0.2592 on 6380 degrees of freedom
## Multiple R-squared:  0.06856,    Adjusted R-squared:  0.06841 
## F-statistic: 469.6 on 1 and 6380 DF,  p-value: < 2.2e-16
\end{verbatim}

Vidimo da naš model objašnjava oko 7\% ukupne varijance u povratu. //
provjeriti recenicu

U nastavku prikazujemo Q-Q plot i graf reziduala

\begin{Shaded}
\begin{Highlighting}[]
\KeywordTok{qqnorm}\NormalTok{(}\KeywordTok{rstandard}\NormalTok{(fit.velicine))}
\KeywordTok{qqline}\NormalTok{(}\KeywordTok{rstandard}\NormalTok{(fit.velicine))}
\end{Highlighting}
\end{Shaded}

\includegraphics{projekt_files/figure-latex/unnamed-chunk-30-1.pdf}

\begin{Shaded}
\begin{Highlighting}[]
\KeywordTok{plot}\NormalTok{(}\KeywordTok{fitted}\NormalTok{(fit.velicine), }\KeywordTok{resid}\NormalTok{(fit.velicine))}
\end{Highlighting}
\end{Shaded}

\includegraphics{projekt_files/figure-latex/unnamed-chunk-30-2.pdf}

\begin{Shaded}
\begin{Highlighting}[]
\NormalTok{matrix_coef <-}\StringTok{ }\KeywordTok{summary}\NormalTok{(fit.velicine)}\OperatorTok{$}\NormalTok{coefficients  }
\NormalTok{my_estimates <-}\StringTok{ }\NormalTok{matrix_coef[ , }\DecValTok{1}\NormalTok{]                   }
\NormalTok{my_estimates }
\end{Highlighting}
\end{Shaded}

\begin{verbatim}
##  (Intercept) log_velicina 
##   0.35277292   0.03896851
\end{verbatim}

S ovim podacima linearne regresije možemo zaključiti da za svako
povećanje od 2.917 * 10**7 USD imovine za upravljanje nekog fonda njegov
povrat se poveća za 0.1\%. // provjeriti rečenicu

Analiziraju se dividende ovisno o stilu investiranja fonda.

\begin{Shaded}
\begin{Highlighting}[]
\NormalTok{returnsComplete =}\StringTok{ }\NormalTok{funds[}\KeywordTok{complete.cases}\NormalTok{(funds[,}\KeywordTok{c}\NormalTok{(}\StringTok{"fund_return_10years"}\NormalTok{,}\StringTok{"category_return_10years"}\NormalTok{,}\StringTok{"fund_sharpe_ratio_10years"}\NormalTok{,}\StringTok{"category_sharpe_ratio_10years"}\NormalTok{,}\StringTok{"years_up"}\NormalTok{,}\StringTok{"years_down"}\NormalTok{)]),]}

\NormalTok{growthFunds =}\StringTok{ }\NormalTok{funds[funds}\OperatorTok{$}\NormalTok{investment }\OperatorTok{==}\StringTok{ "Growth"}\NormalTok{,]}
\NormalTok{valueFunds =}\StringTok{ }\NormalTok{funds[funds}\OperatorTok{$}\NormalTok{investment }\OperatorTok{==}\StringTok{ "Value"}\NormalTok{,]}
\NormalTok{blendFunds =}\StringTok{ }\NormalTok{funds[funds}\OperatorTok{$}\NormalTok{investment }\OperatorTok{==}\StringTok{ "Blend"}\NormalTok{,]}

\KeywordTok{boxplot}\NormalTok{(growthFunds}\OperatorTok{$}\NormalTok{fund_yield,}
\NormalTok{        valueFunds}\OperatorTok{$}\NormalTok{fund_yield,}
\NormalTok{        blendFunds}\OperatorTok{$}\NormalTok{fund_yield, }\DataTypeTok{names=}\KeywordTok{c}\NormalTok{(}\StringTok{"Growth"}\NormalTok{,}\StringTok{"Value"}\NormalTok{,}\StringTok{"Blend"}\NormalTok{), }\DataTypeTok{col=}\KeywordTok{c}\NormalTok{(}\StringTok{"Red"}\NormalTok{, }\StringTok{"Light green"}\NormalTok{, }\StringTok{"Light blue"}\NormalTok{))}
\end{Highlighting}
\end{Shaded}

\includegraphics{projekt_files/figure-latex/unnamed-chunk-33-1.pdf}

Box plot dividendi po stilu investiranja ukazuje na veliku zakrivljenost
kod dividendi Growth fondova. Zbog te činjenice ne možemo koristiti
testove koji se oslanjaju na pretpostavku normalnosti. Za analizu
jednakosti sredina koristit ćemo Kruskal-Wallisov test umjesto ANOVA
testa. Dodatno testiramo razliku dividendi između Value i Blend fondova.
Prije samog testiranja razlike provodimo Kolmogorov-Smirnovljev test
kako bismo utvrdili možemo li za testiranje razlike koristiti t-test ili
moramo koristiti neki od neparametarskih testova. Tvrdnja koju testiramo
je jesu li dividende value fondova veće od dividendi blend fondova

\begin{Shaded}
\begin{Highlighting}[]
\KeywordTok{kruskal.test}\NormalTok{(funds}\OperatorTok{$}\NormalTok{fund_yield}\OperatorTok{~}\NormalTok{funds}\OperatorTok{$}\NormalTok{investment, }\DataTypeTok{data=}\NormalTok{funds);}
\end{Highlighting}
\end{Shaded}

\begin{verbatim}
## 
##  Kruskal-Wallis rank sum test
## 
## data:  funds$fund_yield by funds$investment
## Kruskal-Wallis chi-squared = 1853, df = 3, p-value < 2.2e-16
\end{verbatim}

\begin{Shaded}
\begin{Highlighting}[]
\CommentTok{#jedan od fondova ima različitu dividendu}

\KeywordTok{shapiro.test}\NormalTok{(valueFunds}\OperatorTok{$}\NormalTok{fund_yield)}
\end{Highlighting}
\end{Shaded}

\begin{verbatim}
## 
##  Shapiro-Wilk normality test
## 
## data:  valueFunds$fund_yield
## W = 0.9045, p-value < 2.2e-16
\end{verbatim}

\begin{Shaded}
\begin{Highlighting}[]
\KeywordTok{shapiro.test}\NormalTok{(blendFunds}\OperatorTok{$}\NormalTok{fund_yield)}
\end{Highlighting}
\end{Shaded}

\begin{verbatim}
## 
##  Shapiro-Wilk normality test
## 
## data:  blendFunds$fund_yield
## W = 0.8325, p-value < 2.2e-16
\end{verbatim}

\begin{Shaded}
\begin{Highlighting}[]
\KeywordTok{wilcox.test}\NormalTok{(valueFunds}\OperatorTok{$}\NormalTok{fund_yield, blendFunds}\OperatorTok{$}\NormalTok{fund_yield, }\DataTypeTok{paired =} \OtherTok{FALSE}\NormalTok{, }\DataTypeTok{var.equal =} \OtherTok{FALSE}\NormalTok{, }\DataTypeTok{alternative =} \StringTok{"greater"}\NormalTok{)}
\end{Highlighting}
\end{Shaded}

\begin{verbatim}
## 
##  Wilcoxon rank sum test with continuity correction
## 
## data:  valueFunds$fund_yield and blendFunds$fund_yield
## W = 1916463, p-value < 2.2e-16
## alternative hypothesis: true location shift is greater than 0
\end{verbatim}

Kolmogorov-Smirnovljev testovi ukauzju na činjenicu da fondovi nemaju
normalnu razdiobu te se stoga koristi neparametarski Wilcoxonov test
predznačenih rangova. Navedeni test daje zaključak da value fondovi
imaju veće dividende od blend fondova.

Za istraživanje tvrdnje da fondovi pobjeđuju svoje kategorije,
analizirali smo povrate fondova i njihovih kategorija nakon 10 godina.
Boxplot za povrate fondova i kategorija mogao bi ukazivati na činjenicu
da su povrati fondova i kategorija jednaki s razlikom u većoj varijaciji
kod fondova.

\begin{Shaded}
\begin{Highlighting}[]
\NormalTok{returns_10years =}\StringTok{ }\NormalTok{funds[}\KeywordTok{c}\NormalTok{(}\StringTok{"fund_return_10years"}\NormalTok{,}\StringTok{"category_return_10years"}\NormalTok{)]}
\NormalTok{ind =}\StringTok{ }\KeywordTok{which}\NormalTok{(}\OperatorTok{!}\KeywordTok{is.na}\NormalTok{(returns_10years}\OperatorTok{$}\NormalTok{fund_return_10years) }\OperatorTok{&}\StringTok{ }\OperatorTok{!}\KeywordTok{is.na}\NormalTok{(returns_10years}\OperatorTok{$}\NormalTok{category_return_10years))}
\NormalTok{returns_10years =}\StringTok{ }\NormalTok{returns_10years[ind,]}

\KeywordTok{boxplot}\NormalTok{(funds}\OperatorTok{$}\NormalTok{fund_return_10years, funds}\OperatorTok{$}\NormalTok{category_return_10years, }\DataTypeTok{names=}\KeywordTok{c}\NormalTok{(}\StringTok{"Fund return"}\NormalTok{,}\StringTok{"Category return"}\NormalTok{), }\DataTypeTok{col =} \KeywordTok{c}\NormalTok{(}\StringTok{"light blue"}\NormalTok{,}\StringTok{"yellow"}\NormalTok{))}
\end{Highlighting}
\end{Shaded}

\includegraphics{projekt_files/figure-latex/unnamed-chunk-35-1.pdf}
Analizom histograma povrata fondova i kategorija uočavaju se blage
zakrivljenosti podataka, ali u histogramu razlike povrata fonda i
povrata kategorije ne uočava se veća zakrivljenost podataka. Za analizu
razlike sredina koristit ćemo upareni t-test jer svaki fond ima
pridruženu odgovarajuću kategoriju. T-test je robustan na manje
zakrivljenosti u podacima te je bolja opcija u odnosu na neparametarske
testove (pretežno zbog svoje veličine) kod velikih uzoraka. Nulta
hipoteza testa je da fondovi imaju manji ili jednak povrat od svojih
kategorija. Alternativna hipoteza testa je da fondovi imaju veći povrat
od svojih kategorija.

\begin{Shaded}
\begin{Highlighting}[]
\KeywordTok{hist}\NormalTok{(returns_10years}\OperatorTok{$}\NormalTok{fund_return_10years, }\DataTypeTok{main=}\StringTok{"Fund return (10 years)"}
\NormalTok{     ,}\DataTypeTok{col=}\StringTok{"cyan"}
\NormalTok{     ,}\DataTypeTok{xlab =} \StringTok{"Fund return"}\NormalTok{)}
\end{Highlighting}
\end{Shaded}

\includegraphics{projekt_files/figure-latex/unnamed-chunk-36-1.pdf}

\begin{Shaded}
\begin{Highlighting}[]
\KeywordTok{hist}\NormalTok{(returns_10years}\OperatorTok{$}\NormalTok{category_return_10years, }\DataTypeTok{main=}\StringTok{"Category return (10 years)"}\NormalTok{, }\DataTypeTok{col=}\StringTok{"red"}\NormalTok{, }\DataTypeTok{xlab =} \StringTok{"Category return"}\NormalTok{)}
\end{Highlighting}
\end{Shaded}

\includegraphics{projekt_files/figure-latex/unnamed-chunk-36-2.pdf}

\begin{Shaded}
\begin{Highlighting}[]
\KeywordTok{hist}\NormalTok{(returns_10years}\OperatorTok{$}\NormalTok{fund_return_10years }\OperatorTok{-}\StringTok{ }\NormalTok{returns_10years}\OperatorTok{$}\NormalTok{category_return_10years, }\DataTypeTok{main=}\StringTok{"Fund - category return (10 years)"}\NormalTok{, }\DataTypeTok{col=}\StringTok{"purple"}\NormalTok{, }\DataTypeTok{xlab=}\StringTok{"Fund - category return"}\NormalTok{)}
\end{Highlighting}
\end{Shaded}

\includegraphics{projekt_files/figure-latex/unnamed-chunk-36-3.pdf}

\begin{Shaded}
\begin{Highlighting}[]
\KeywordTok{print}\NormalTok{(}\StringTok{"Wilcoxonov test nad razlikom povrata fonda i povrata kategorije u razdoblju od 10 godina."}\NormalTok{)}
\end{Highlighting}
\end{Shaded}

\begin{verbatim}
## [1] "Wilcoxonov test nad razlikom povrata fonda i povrata kategorije u razdoblju od 10 godina."
\end{verbatim}

\begin{Shaded}
\begin{Highlighting}[]
\KeywordTok{t.test}\NormalTok{(}\DataTypeTok{x =}\NormalTok{ funds}\OperatorTok{$}\NormalTok{fund_return_10years, }
            \DataTypeTok{y =}\NormalTok{ funds}\OperatorTok{$}\NormalTok{category_return_10years, }
            \DataTypeTok{paired =} \OtherTok{TRUE}\NormalTok{,}
            \DataTypeTok{alternative =} \StringTok{"greater"}\NormalTok{, }
            \DataTypeTok{conf.level =} \FloatTok{0.99}\NormalTok{)}
\end{Highlighting}
\end{Shaded}

\begin{verbatim}
## 
##  Paired t-test
## 
## data:  funds$fund_return_10years and funds$category_return_10years
## t = -8.8946, df = 6385, p-value = 1
## alternative hypothesis: true difference in means is greater than 0
## 99 percent confidence interval:
##  -0.3032769        Inf
## sample estimates:
## mean of the differences 
##              -0.2403883
\end{verbatim}

Zaključak testiranja je da ne možemo odbaciti tvrdnju da fondovi imaju
manji ili jednak povrat od svojih kategorija, odnosno nismo uspijeli
dokazati da fondovi imaju veći povrat od svojih kategorija.

Linearnom regresijom provjerit će se ovisi li povrat fonda o povratu
kategorije. Drugim riječima analizira se prate li fondovi svoje
kategorije u smislu povrata.

\begin{Shaded}
\begin{Highlighting}[]
\NormalTok{returnsComplete =}\StringTok{ }\NormalTok{funds[}\KeywordTok{complete.cases}\NormalTok{(}\KeywordTok{c}\NormalTok{(}\StringTok{"fund_return_10years"}\NormalTok{,}\StringTok{"category_return_10years"}\NormalTok{)),]}
\NormalTok{x =}\StringTok{ }\NormalTok{returnsComplete}\OperatorTok{$}\NormalTok{category_return_10years}
\NormalTok{y =}\StringTok{ }\NormalTok{returnsComplete}\OperatorTok{$}\NormalTok{fund_return_10years}

\NormalTok{fit.return =}\StringTok{ }\KeywordTok{lm}\NormalTok{(y}\OperatorTok{~}\NormalTok{x, }\DataTypeTok{data=}\NormalTok{returnsComplete)}
\KeywordTok{plot}\NormalTok{(x, y, }\DataTypeTok{main =} \StringTok{"Category and fund returns regression"}\NormalTok{, }\DataTypeTok{xlab=}\StringTok{"Category return (10 years)"}\NormalTok{, }\DataTypeTok{ylab=}\StringTok{"Fund return (10 years)"}\NormalTok{)}
\KeywordTok{lines}\NormalTok{(x, fit.return}\OperatorTok{$}\NormalTok{fitted.values, }\DataTypeTok{col=}\StringTok{"red"}\NormalTok{)}
\end{Highlighting}
\end{Shaded}

\includegraphics{projekt_files/figure-latex/unnamed-chunk-37-1.pdf}
Uočava se potencijalna zavisnost povrata fonda o povratu kategorije te
je potrebno provesti analizu reziduala. Također će se odrediti Pearsonov
koeficijent korelacije koji će predstalvljati jačinu linearne veze, kao
i koeficijent determinacije koji određuje kvalitetu modela

Analiza reziduala:

\begin{Shaded}
\begin{Highlighting}[]
\KeywordTok{qqnorm}\NormalTok{(}\KeywordTok{rstandard}\NormalTok{(fit.return))}
\KeywordTok{qqline}\NormalTok{(}\KeywordTok{rstandard}\NormalTok{(fit.return))}
\end{Highlighting}
\end{Shaded}

\includegraphics{projekt_files/figure-latex/unnamed-chunk-38-1.pdf}

\begin{Shaded}
\begin{Highlighting}[]
\KeywordTok{plot}\NormalTok{(}\KeywordTok{fitted}\NormalTok{(fit.return), }\KeywordTok{resid}\NormalTok{(fit.return))}
\KeywordTok{abline}\NormalTok{(}\DecValTok{0}\NormalTok{,}\DecValTok{0}\NormalTok{)}
\end{Highlighting}
\end{Shaded}

\includegraphics{projekt_files/figure-latex/unnamed-chunk-38-2.pdf}

\begin{Shaded}
\begin{Highlighting}[]
\CommentTok{#c("Koeficijent determinacije: ",rsq(x,y))}
\KeywordTok{c}\NormalTok{(}\StringTok{"Pearsonov koeficijent korelacije:"}\NormalTok{, }\KeywordTok{cor}\NormalTok{(x,y,}\DataTypeTok{method=}\StringTok{"pearson"}\NormalTok{))}
\end{Highlighting}
\end{Shaded}

\begin{verbatim}
## [1] "Pearsonov koeficijent korelacije:" "0.808657328513655"
\end{verbatim}

\begin{Shaded}
\begin{Highlighting}[]
\KeywordTok{summary}\NormalTok{(fit.return)}
\end{Highlighting}
\end{Shaded}

\begin{verbatim}
## 
## Call:
## lm(formula = y ~ x, data = returnsComplete)
## 
## Residuals:
##     Min      1Q  Median      3Q     Max 
## -32.508  -0.923   0.189   1.223  10.870 
## 
## Coefficients:
##              Estimate Std. Error t value Pr(>|t|)    
## (Intercept) -0.157373   0.125952  -1.249    0.212    
## x            0.993893   0.009049 109.831   <2e-16 ***
## ---
## Signif. codes:  0 '***' 0.001 '**' 0.01 '*' 0.05 '.' 0.1 ' ' 1
## 
## Residual standard error: 2.16 on 6384 degrees of freedom
## Multiple R-squared:  0.6539, Adjusted R-squared:  0.6539 
## F-statistic: 1.206e+04 on 1 and 6384 DF,  p-value: < 2.2e-16
\end{verbatim}

\begin{Shaded}
\begin{Highlighting}[]
\KeywordTok{coefficients}\NormalTok{(fit.return)}
\end{Highlighting}
\end{Shaded}

\begin{verbatim}
## (Intercept)           x 
##  -0.1573729   0.9938934
\end{verbatim}

Pearsonov koeficijent i koeficijent determinacije ukazuju na jaku
linearnu vezu između povrata kategorije i povrata fonda. Reziduali ne
pokazuju nikakvu vezu što je nužno za provođenje zaključaka o linearnog
regresiji. Zaključuje se da postoji jaka linearna veza između povrata
fondova i povrata njihovih kategorija. Procjenjeni koeficijenti iznose:
b1 = 0.993893 i b0 = -0.157373. Ovom regresijom ne možemo donositi
zaključke pobjeđuje li fond svoju kategoriju, već samo donosimo
zaključak da fondovi prate svoje kategorije u smislu povrata.

Za provjere hipoteze o pobjeđivanju fondova analizirati će se zavisnost
razlike povrata fonda i povrata kategorije o povratu kateogrije.

\begin{Shaded}
\begin{Highlighting}[]
\NormalTok{returnsComplete =}\StringTok{ }\NormalTok{funds[}\KeywordTok{complete.cases}\NormalTok{(}\KeywordTok{c}\NormalTok{(}\StringTok{"fund_return_10years"}\NormalTok{,}\StringTok{"category_return_10years"}\NormalTok{)),]}
\NormalTok{x =}\StringTok{ }\NormalTok{returnsComplete}\OperatorTok{$}\NormalTok{category_return_10years}
\NormalTok{y =}\StringTok{ }\NormalTok{returnsComplete}\OperatorTok{$}\NormalTok{fund_return_10years }\OperatorTok{-}\StringTok{ }\NormalTok{returnsComplete}\OperatorTok{$}\NormalTok{category_return_10years}

\NormalTok{fit.return =}\StringTok{ }\KeywordTok{lm}\NormalTok{(y}\OperatorTok{~}\NormalTok{x, }\DataTypeTok{data=}\NormalTok{returnsComplete)}
\KeywordTok{plot}\NormalTok{(x, y, }\DataTypeTok{xlab=}\StringTok{"Category return (10 years)"}\NormalTok{, }\DataTypeTok{ylab=}\StringTok{"Fund - category return (10 years)"}\NormalTok{)}
\KeywordTok{lines}\NormalTok{(x, fit.return}\OperatorTok{$}\NormalTok{fitted.values, }\DataTypeTok{col=}\StringTok{"red"}\NormalTok{)}
\end{Highlighting}
\end{Shaded}

\includegraphics{projekt_files/figure-latex/unnamed-chunk-39-1.pdf}
Uočava se vrlo slaba linearna zavisnost između razlike povrata fonda i
kategorije te same kategorije.

Analiza reziduala:

\begin{Shaded}
\begin{Highlighting}[]
\KeywordTok{qqnorm}\NormalTok{(}\KeywordTok{rstandard}\NormalTok{(fit.return))}
\KeywordTok{qqline}\NormalTok{(}\KeywordTok{rstandard}\NormalTok{(fit.return))}
\end{Highlighting}
\end{Shaded}

\includegraphics{projekt_files/figure-latex/unnamed-chunk-40-1.pdf}

\begin{Shaded}
\begin{Highlighting}[]
\KeywordTok{plot}\NormalTok{(}\KeywordTok{fitted}\NormalTok{(fit.return), }\KeywordTok{resid}\NormalTok{(fit.return))}
\KeywordTok{abline}\NormalTok{(}\DecValTok{0}\NormalTok{,}\DecValTok{0}\NormalTok{)}
\end{Highlighting}
\end{Shaded}

\includegraphics{projekt_files/figure-latex/unnamed-chunk-40-2.pdf}

\begin{Shaded}
\begin{Highlighting}[]
\CommentTok{#c("Koeficijent determinacije: ",rsq(x,y))}
\KeywordTok{c}\NormalTok{(}\StringTok{"Pearsonov koeficijent korelacije:"}\NormalTok{, }\KeywordTok{cor}\NormalTok{(x,y,}\DataTypeTok{method=}\StringTok{"pearson"}\NormalTok{))}
\end{Highlighting}
\end{Shaded}

\begin{verbatim}
## [1] "Pearsonov koeficijent korelacije:" "-0.00844555250251895"
\end{verbatim}

\begin{Shaded}
\begin{Highlighting}[]
\KeywordTok{summary}\NormalTok{(fit.return)}
\end{Highlighting}
\end{Shaded}

\begin{verbatim}
## 
## Call:
## lm(formula = y ~ x, data = returnsComplete)
## 
## Residuals:
##     Min      1Q  Median      3Q     Max 
## -32.508  -0.923   0.189   1.223  10.870 
## 
## Coefficients:
##              Estimate Std. Error t value Pr(>|t|)
## (Intercept) -0.157373   0.125952  -1.249    0.212
## x           -0.006107   0.009049  -0.675    0.500
## 
## Residual standard error: 2.16 on 6384 degrees of freedom
## Multiple R-squared:  7.133e-05,  Adjusted R-squared:  -8.53e-05 
## F-statistic: 0.4554 on 1 and 6384 DF,  p-value: 0.4998
\end{verbatim}

\begin{Shaded}
\begin{Highlighting}[]
\KeywordTok{coefficients}\NormalTok{(fit.return)}
\end{Highlighting}
\end{Shaded}

\begin{verbatim}
##  (Intercept)            x 
## -0.157372861 -0.006106648
\end{verbatim}

Pearsonov koeficijent korelacije te koeficijent determinacije svojim
niskim vrijednostima ukazuju na gotovo nikakvu linearnu zavisnost između
razlike povrata fonda i kategorije i povrata kategorije. Koeficijenti
procijenjenog pravca su isto približno jednaki nuli što ukazuje na
činjenicu da će razlika povrata fonda i kategorije unutar svake
kategorije ravnomjerno varirati, neovisno o povratu kategorije.

Višestrukom regresijom pokušati ćemo pronaći kauzalnu vezu između
razlike povrata fonda i kategorije i drugih pokazatelja fonda. Parametri
koji bi mogli utjecati na razliku povrata fonda i povrata kategorije su:
ukupna imovina pod upravljanje (pretpostavljamo da fondovi s većom
imovinom pod upravljanje imaju veći vjerojatnost pobijediti svoje
kategorije), razlika godišnjeg troška upravljanja fonda i kategorije
(fondovi koji uzimaju veći postotak za upravljanje fondom imaju veću
vjerojatnost pobijediti svoju kategoiju), medijalna tržišna
kapitalizacija (fondovi koji ulažu u tvrtke s većom tržišnom
kapitalizacijom će vjerojatnije pobijediti svoje kategorije).

\begin{Shaded}
\begin{Highlighting}[]
\NormalTok{returnsComplete =}\StringTok{ }\NormalTok{funds[}\KeywordTok{complete.cases}\NormalTok{(funds[,}\KeywordTok{c}\NormalTok{(}\StringTok{"fund_return_10years"}\NormalTok{,}\StringTok{"category_return_10years"}\NormalTok{,}\StringTok{"fund_sharpe_ratio_10years"}\NormalTok{,}\StringTok{"category_sharpe_ratio_10years"}\NormalTok{,}\StringTok{"years_up"}\NormalTok{,}\StringTok{"net_assets"}\NormalTok{)]),]}

\NormalTok{x1 =}\StringTok{ }\KeywordTok{log}\NormalTok{(returnsComplete}\OperatorTok{$}\NormalTok{net_assets)}
\NormalTok{x2 =}\StringTok{ }\NormalTok{returnsComplete}\OperatorTok{$}\NormalTok{net_annual_expense_ratio_fund }\OperatorTok{-}\StringTok{ }\NormalTok{returnsComplete}\OperatorTok{$}\NormalTok{net_annual_expense_ratio_category}
\NormalTok{x3 =}\StringTok{ }\NormalTok{returnsComplete}\OperatorTok{$}\NormalTok{median_market_cap}

\KeywordTok{cor}\NormalTok{(}\KeywordTok{cbind}\NormalTok{(x1,x2,x3))}
\end{Highlighting}
\end{Shaded}

\begin{verbatim}
##            x1          x2          x3
## x1  1.0000000 -0.29043724  0.18342861
## x2 -0.2904372  1.00000000 -0.04490357
## x3  0.1834286 -0.04490357  1.00000000
\end{verbatim}

Analiziranjem koreliranosti varijabli primjećuje se slaba koreliranost
između varijabli

\begin{Shaded}
\begin{Highlighting}[]
\NormalTok{y =}\StringTok{ }\NormalTok{returnsComplete}\OperatorTok{$}\NormalTok{fund_return_10years }\OperatorTok{-}\StringTok{ }\NormalTok{returnsComplete}\OperatorTok{$}\NormalTok{category_return_10years}


\NormalTok{fit.return =}\StringTok{ }\KeywordTok{lm}\NormalTok{(y}\OperatorTok{~}\NormalTok{x1 }\OperatorTok{+}\StringTok{ }\NormalTok{x2 }\OperatorTok{+}\StringTok{ }\NormalTok{x3, }\DataTypeTok{data=}\NormalTok{returnsComplete)}
\KeywordTok{summary}\NormalTok{(fit.return)}
\end{Highlighting}
\end{Shaded}

\begin{verbatim}
## 
## Call:
## lm(formula = y ~ x1 + x2 + x3, data = returnsComplete)
## 
## Residuals:
##      Min       1Q   Median       3Q      Max 
## -29.6037  -0.8129   0.0828   0.9971  11.1332 
## 
## Coefficients:
##               Estimate Std. Error t value Pr(>|t|)    
## (Intercept) -6.845e+00  2.961e-01 -23.114  < 2e-16 ***
## x1           3.311e-01  1.459e-02  22.695  < 2e-16 ***
## x2          -1.162e+00  5.110e-02 -22.745  < 2e-16 ***
## x3          -2.256e-06  6.608e-07  -3.413 0.000646 ***
## ---
## Signif. codes:  0 '***' 0.001 '**' 0.01 '*' 0.05 '.' 0.1 ' ' 1
## 
## Residual standard error: 1.95 on 6066 degrees of freedom
## Multiple R-squared:  0.1932, Adjusted R-squared:  0.1928 
## F-statistic: 484.2 on 3 and 6066 DF,  p-value: < 2.2e-16
\end{verbatim}

Uočava se vrlo slaba linearna zavisnost između razlike povrata fonda i
kategorije te same samih parametara.

\end{document}
